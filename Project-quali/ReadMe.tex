\documentclass{article}
\usepackage{graphicx}
\usepackage{amsmath}

\title{Genetic Programming Project Documentation}
\author{Atiyeh Sayadi, Baran Shajari}
\date{}


\begin{document}

\maketitle

\section{Project Overview}
This project consists of four Python code files and multiple text files, each containing 1 to 100 matrices generated by \texttt{generate\_matrix.py}. The matrices are randomly generated in different sizes, such as 4\(\times\)4, to test the project under varying conditions.

\section{Code Files Description}
\begin{itemize}
    \item \texttt{generate\_matrix.py}: This file generates random matrices and saves them in text files to be used for testing the algorithm at different matrix sizes.
    \item \texttt{main\_gp.py}: This file contains all the Genetic Programming (GP) algorithms developed for the project. We implemented GP from scratch without using any external libraries.
    \item \texttt{read\_matrices.py}: This file reads matrices from the generated text files or from existing matrices, utilizes the functions in \texttt{main\_gp.py}, runs the GP algorithm, and records the number of generations needed to reduce inconsistency below 0.3 for each matrix. Additionally, it generates a plot to show the fitness function (inconsistency) trend based on generations for the first matrix.
    \item \texttt{bar\_chart.py}: This file creates bar charts that display the number of matrices that reach consistency within specific numbers of generations.
\end{itemize}

\section{Execution Instructions}
To run the project, follow these steps:
\begin{enumerate}
    \item Generate random matrices using \texttt{generate\_matrix.py} or use the existing matrices.
    \item Run \texttt{read\_matrices.py} to read the matrices, execute the GP algorithm, and observe the generations needed to achieve a consistency threshold of 0.3.
    \item Execute \texttt{bar\_chart.py} to visualize the distribution of generations needed for different matrices to reach consistency.
\end{enumerate}

\section{Output and Visualizations}
\begin{itemize}
    \item \texttt{read\_matrices.py} provides a plot showing the fitness function trend for the first matrix, illustrating how inconsistency decreases with each generation.
    \item \texttt{bar\_chart.py} generates bar charts that represent the count of matrices reaching consistency within specific generation ranges.
\end{itemize}

\end{document}

